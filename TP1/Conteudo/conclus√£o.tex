\section{Conclusão}

    Ao longo da realização deste trabalho prático enfrentámos vários desafios que nos obrigaram a pensar com espírito crítico e a refletir sobre quais seriam as melhores estratégias a adotar em cada situação, nomeadamente na definição de cabeçalhos e na arquitetura dos \textit{end systems}.

    Voltando atrás e analisando todo o caminho percorrido, pensamos que teria sido uma mais-valia aplicar \textit{bind9} no projeto, visto que isso tornaria o sistema relativamente mais robusto e flexível. Assim sendo, no futuro, desejamos vir a explorar mais detalhadamente este serviço, pois claramente muitos pontos ficaram por esclarecer. 

    Em suma, e apesar dalguns contratempos, julgámos que este trabalho prático foi útil no sentido em que consolidou alguns conceitos que tinham ficado pouco claros nas aulas teóricas, e além disso, abriu-nos os olhos para a complexidade exigida no nível aplicacional quando se utiliza o UDP enquanto protocolo da camada transporte.