\section{Conclusão}

    Quando iniciámos a realização deste trabalho prático desconhecíamos por completo o comportamento dos protocolos da camada de transporte, tampouco sabíamos as aplicações que utilizavam cada um. Todavia, graças às aulas teóricas e várias pesquisas nas \textit{internet,} todas as nossas dúvidas foram satisfeitas.

    A escolha efetuada para a utilização de um protocolo de transporte tem de ser muito bem ponderada, pois ao darmos prioridade a determinado comportamento, acabamos por perder diversas funcionalidades ou características que poderiam ser-nos úteis no futuro. Assim sendo, verificámos que grande parte das aplicações que possuem algum nível de segurança dão prioridade o TCP, enquanto que as restantes adquirem velocidade ao utilizar UDP.

    É certo que o \textit{handshake} inerente ao TCP atrasa bastante as conexões, contudo é isso que permite aos \textit{end systems} saberem que do outro lado há alguém disposto a receber/enviar mensagens.

    Em suma, este trabalho foi bastante proveitoso na medida em que contribuiu para aprofundar os nossos conhecimentos relativos aos protocolos, e ainda explorar algumas características e funcionalidades das aplicações. 